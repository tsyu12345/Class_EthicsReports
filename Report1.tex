\documentclass{article}[jsarticle]
\usepackage[T1]{fontenc}
\usepackage[dvipdfmx]{hyperref}
\usepackage{lmodern}
\usepackage{latexsym}
\usepackage{amsfonts}
\usepackage{amssymb}
\usepackage{mathtools}
\usepackage{nccmath}
\usepackage{amsthm}
\usepackage{multirow}
\usepackage{graphicx}
\usepackage[dvipdfmx]{color}
\usepackage{wrapfig}
\usepackage{here}
\usepackage{float}
\usepackage{ascmac}
\usepackage{url}
\usepackage{listings}
\usepackage{xcolor}
\usepackage{pifont}
\usepackage{hyperref}

\title{先端科学技術の倫理 個人課題}
\author{高林秀 \\ 三宅研究室 博士前期課程1年 \\ V-CampusID : 23vr008n}
\date{\today}

\begin{document}

\maketitle

\begin{abstract}
    本稿は本年度必修授業の先端科学技術の倫理の個人課題レポートである。\par
\end{abstract}

\tableofcontents

\section{課題1}
\subsection{課題内容}
以下の文章の要旨をまとめる。\par
\href{https://nvlpubs.nist.gov/nistpubs/SpecialPublications/NIST.SP.800-171r2.pdf}{Protecting Controlled Unclassified Information in Nonfederal Systems and Organizations}

\subsection{解答}
この論文は、米国連邦政府(以下、連邦政府)に対し、非連邦システムと組織の非機密情報を保護するためのセキュリティ要件を提示するものである。\par
本論文が提供する要件は以下の場合において適用されるものである。
\begin{enumerate}
    \item CUIが連邦政府以外のシステムや組織に常駐している場合
    \item 連邦政府以外の組織が、連邦政府機関に代わってCUIを収集または管理していない場合、または連邦政府機関に代わってシステムを使用または運用していない場合。
    \item CUIレジストリに記載されているCUIカテゴリについて、権限を付与する法律、規則、または連邦政府全体の方針によって規定される保護要件がない場合
\end{enumerate}
なお本論文は、連邦政府機関とそれ以外の組織との間で締結される契約書や、その他の合意において、連邦政府機関が使用することを意図するものである。
\subsubsection{第1章:本論文の概要・構成と対象者について}

\subsubsection{第2章}
\subsubsection{第3章}

\section{課題2}
\subsection{課題内容}
\subsection{解答}

\section{課題3}
\subsection{課題内容}
\subsection{解答}

\end{document}
